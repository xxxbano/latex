% use exlatex command to complie this file

\documentclass[11pt]{article}
\usepackage{fontspec,xunicode,xltxtra}
\usepackage{amsmath}
\usepackage{titlesec}
\usepackage{ragged2e}
\usepackage{rotating}
\usepackage{pbox}
\usepackage{hyperref}
\usepackage[table]{xcolor}
%\usepackage{color,colortbl}
\usepackage[top=1in,bottom=1in,left=1.25in,right=1.25in]{geometry}

\hypersetup{ 
	colorlinks=true,
	linkcolor=blue,
	urlcolor=red,
	linktoc=all
}

\definecolor{Gray}{gray}{0.9}

\XeTeXlinebreaklocale "ja"
\XeTeXlinebreakskip = 0pt plus 1pt minus 0.1pt

%======Fonts list======================
%==font size [tiny,scriptsize,footnotesize,small,normalsize,large,Large,LARGE,huge,Huge]
%======================================
\newfontfamily\msg{MS Gothic}
\newfontfamily\mspg{MS PGothic}
\newfontfamily\msm{MS Mincho}
\newfontfamily\mspm{MS PMincho}
%\newfontfamily\msjhei{Microsoft JhengHei}
\setmainfont{MS Mincho}
%\renewcommand{\baseinestretch}{1.25}

%=====Title setting==================
%==\part,\chapter,\section,\subsection,\subsubsection,\paragraph,\subparagraph
%==shape:hang,block,display,runin,leftmargin,rightmargin,drop,wrap,frame
%====================================
\titleformat{\section}[hang]{\large\msg}{\thesection}{0.5em}{}
\titlespacing{\section}{0pc}{0.5pc}{0.3pc}

\titleformat{\subsection}[hang]{\normalsize\msg}{\thesubsection}{0.5em}{}
\titlespacing{\subsection}{0pc}{0.5pc}{0.3pc}

\titleformat{\paragraph}[runin]{\normalsize\msg}{\theparagraph}{0em}{}
\titlespacing{\paragraph}{0pc}{1pc}{0.5pc}

\renewcommand{\baselinestretch}{1.4}
\renewcommand{\contentsname}{目次}
\renewcommand{\refname}{参考文献}
\renewcommand{\tablename}{表}
\renewcommand{\figurename}{図}
%\renewcommand{\appandixname}{付録}


\title{
\LARGE\msg title 
}
\author{
\normalsize company 
\\[-0.5pc]\normalsize R&D 
}
%\date{\normalsize \today}

\begin{document}

%\maketitle

\tableofcontents

\thispagestyle{empty}

\newpage
\section{はじめに}
\setcounter{page}{1}

急速なネットワーク環境の充実に伴い、

\section{耐タンパ}

\subsection{ブロック図}

\begin{figure}[!h]
\centering
\includegraphics[width=6cm]{./fig/overall.eps}
\vspace{-13pt}
\caption{コアブロック図}
\label{fig:overall}
\end{figure}

コアブロック図を図\ref{fig:overall}に示す。

\subsection{性能}

\begin{table}[!h]
\vspace{-13pt}
\caption{性能}
\label{tab:performance}
\centering
\vspace{5pt}
\begin{tabular}{|l|l|}
\hline
長 & 128-bit  \\
\hline
大 & 170 MHz  \\
\hline
規模 & 6890 gate \\
\hline
レイテンシ & 230 clk \\
\hline
速度 & 94 Mbps \\
\hline
\end{tabular}
\end{table}

\subsection{説明}

\begin{table}[!h]
%\vspace{-13pt}
%\caption{説明}
%\label{tab:t}
\centering
\vspace{5pt}
\begin{tabular}{|l|c|p{10.2cm}|}
\rowcolor{Gray}
\hline
信号名 & IN/OUT & 説明 \\
\hline
clk & IN & クロック \\
\hline
asrstB & IN & 非同期リセット(Active low)\\
\hline
ctl [3:0] & IN & 
制御信号 \newline 
ctl[3:0]=4'b0000、選択してない。 \newline
ctl[3:0]=4'b0001、入力する。 \newline
ctl[3:0]=4'b0010、入力する。 \newline
ctl[3:0]=4'b0011、入力する。 \newline
ctl[3:0]=4'b0100、入力して。 \newline
ctl[3:0]=4'b0101、入力して。 \newline
ctl[3:0]=4'b0110、入力して。\newline
Others、定義なし。\\
\hline
ibs [31:0] & IN &
バス入力信号 \\
\hline
ird [7:0] & IN &
入力信号 \\
\hline
srd & IN &
入力信号 \\
\hline
ack & OUT &
通知信号 \newline
ack=0、ビジー状態(操作できない)。 \newline
ack=1、操作できる状態。 \\
\hline
obs [31:0] & OUT &
バス出力  \\
\hline
\end{tabular}
\end{table}

\subsection{使用説明}
\label{sec:ip}

\begin{figure}[!h]
\centering
\includegraphics[width=11cm]{./fig/enc.eps}
\vspace{-13pt}
\caption{入出力順番}
\label{fig:enc}
\end{figure}

\begin{figure}[!h]
\centering
\includegraphics[width=11cm]{./fig/dec.eps}
\vspace{-13pt}
\caption{出力順番}
\label{fig:dec}
\end{figure}

異なる。

詳しい処理順番を以下に記す。
\vspace{-1ex} \begin{enumerate}
\setlength{\itemsep}{-1ex}
%\renewcommand\theenumi{\textcircled{\scriptsize\arabic{enumi}}} 
%\renewcommand\theenumi{\textcircled{\footnotesize\arabic{enumi}}} 
%\renewcommand\theenumi{\textcircled{\arabic{enumi}}} 
%\renewcommand{\labelenumi}{\theenumi}
\item リセット。
\item 鍵入力。
\end{enumerate}

\subsection{信号}
\ref{sec:ip}で述べた

出力順番


%\section{まとめ}


\begin{thebibliography}{9}

\bibitem{exp} "組みガイド," 機構, 2000年3月.

\end{thebibliography}

\begin{appendix}
\end{appendix}


\end{document}             % End of documentp
